\input{src/macros.tex}

\author{
    Marc \textsc{Chevalier}
}
\date{\today}
\title{Computational complexity \--- Homework}

\begin{document}

\maketitle

\section{NP-Hardness}
\subsection{Halting problem}

Let be $\varphi$ an instance of SAT problem. We denote by $n$ the number of variables.

Let be $M$ a \textsc{Turing} machine which tests in a cycle all the $2^n$ possible assignations of the previous formula : when $M$ has tested all assignations, it starts again. This machine halts if and only if $\varphi$ is satisfiable. This reduction is polynomial, therefore $SAT \leqslant_p \textsc{Halt}$, ie. $\textsc{Halt}$ is $NP$-hard since $SAT$ is $NP$-hard.

$\textsc{Halt}$ is not $NP$-complete otherwise it was decidable by a \textsc{Turing}-machine, but $\textsc{Halt}$ is undecidable.

\begin{theorem}
    \textsc{Halt} is $\NP$-hard but not $\NP$-complete
\end{theorem}

\subsection{\texorpdfstring{$TQBF$}{TQBF}}

All instance of $SAT$ problem is an instance of $TQBF$. Without transformation, we have a polynomial reduction, ie. $SAT\leqslant_p TQBF$ so $TQBF$ is $NP$-hard.

This problem is known for being $\pSpace$-complete. It's maybe $\NP$-complete.

\begin{theorem}
    $TQBF$ is $\NP$-hard.
\end{theorem}
   
\subsection{\texorpdfstring{$NAE-3-SAT$}{NAE-3-SAT}}

Let be $\varphi$ an instance of $NAE-3-SAT$.

$$
    \varphi = \bigwedge\limits_{i=1}^n \left(x_{i,1}^* \vee x_{i,2}^* \vee x_{i,3}^*\right)
$$
where $x_{i,j}^*$ is the literal $x_{i,j}$ or $\neg x_{i,j}$.

We will describe the "not all equal" condition is term of formula.

$$
    \bigwedge\limits_{i=1}^n \neg(x_{i,1}^* \wedge x_{i,2}^* \wedge x_{i,3})^* \wedge \neg(\neg x_{i,1}^* \wedge \neg x_{i,2}^* \wedge \neg x_{i,3}^*)
$$
using \textsc{De Morgan}'s law:
$$
    \psi:=\bigwedge\limits_{i=1}^n (\neg x_{i,1}^* \vee \neg x_{i,2}^* \vee \neg x_{i,3}^*) \wedge (x_{i,1}^* \vee x_{i,2}^* \vee x_{i,3}^*)
$$

$\vert \psi\vert \sim 2\vert \varphi \vert \Rightarrow \vert \psi \vert = \cplx{\vert\varphi\vert}$. 

Let $\varpi := \varphi \wedge \psi$. $\varpi$ is an instance of $SAT$ and the reduction is polynomial. If there is an solution to the $SAT$ problem $\xi$, then $\varphi$ is satisfied and, thanks to $\psi$, the "not all equal" condition is true.

Reciprocally, if there is a solution of the $NAE-3-SAT$ problem $\varphi$, then this assignation makes $\xi$ true.

Consequently, $SAT \leqslant_p NAE-3-SAT$ and $NAE-3-SAT$ is $NP$-hard. 

Moreover, $NAE-3-SAT$ is clearly $NP$: a valid assignation is a sufficient witness. We can check in polynomial time if this assignation makes the formula true and if the "not all equal" condition is satisfied. Then $NAE-3-SAT$ is $NP$-complete.

\begin{theorem}
    $NAE-3-SAT$ is $\NP$-complete
\end{theorem}
   

\subsection{\textsc{MaxCut}}

Let $F$ be an instance of $NAE-3-SAT$
$$
    F=\bigwedge\limits_{i=1}^m C_i
$$

We produce a graph $G=(V,E)$ which has a vertex for each literal of $F$. There is a edge between two vertices if there is a clause which contains this two literals. So each clause is described by a triangle. Moreover, we add $\vert F\vert_{x_i}$ (the number of occurrences of $x_i$ in $F$) edges between $x_i$ and $\neg x_i$. The size of the cut we search is at least $5m$.

If we have an assignment of the $NAE-3-SAT$, we take the vertices which are true in $S$ and the other in $\overline{S}$. So, we have $2m$ from the triangles due to the clauses and $3m$ from the edges between all pair $(x_i,\neg x_i)$.

Reciprocally, if we have a cut of size $\geqslant 5m$.

If we have no pair $(x_i, \neg x_i)$ on the same side, we have a valid assignment.

If there is a such pair, we can move one of them on the opposite side without decreasing the size of the cut. Let $n_i$ the number of edges between $x_i$ and $\neg x_i$. We note $a$ the number of edges which $x_i$ is an extremity and which the other is in the opposite side. We note $b$ the number of edges between $b$ and a vertex of the opposite size. We know that $a+b \leqslant 2n_i$. If we move $x_i$ in the opposite size, the cut gains $n_i-a$ edges. If $\neg x_i$ go to the opposite side, it gains $n_i-b$. $\max(n_i-a,n_i-b)\geqslant 0$, so we can move one of these vertices to the opposite side without decreasing the size of the cut. We redo this transformation until we reach a cut of the first case (at most $m$ times).

We proved that $NAE-3-SAT \leqslant_p \textsc{MaxCut}$.

Moreover, $\textsc{MaxCut}$ is in $\NP$. Indeed, a witness is the list of the vertices of $S$ (or $\overline{S}$). The size is actually polynomial with respect of the size of $G$ and we can check the solution in a polynomial time : we check easily that the cut has a size $\geqslant k$ in a quadratic time.

So, $\textsc{MaxCut}\in\NP$.

\begin{theorem}
    $\textsc{MaxCut}$ is $\NP$-complete
\end{theorem}
   

\section{Reductions}
\begin{propo}
    The set of the recursive languages is closed under polynomial-time \textsc{Karp} reduction.
\end{propo}
\begin{proof}
    That is obvious, isn't it ?
\end{proof}
\begin{propo}
    $\textsc{DLogTime}$ is not closed under polynomial-time \textsc{Karp} reduction.
\end{propo}
\begin{proof}
    Let $\L \in \textsc{DLogTime}$ such that $\lvert \L \rvert \geqslant 2$. For all $\L' \in \pTime\setminus\textsc{DLogTime}$ (there is at least one), there is a polynomial time function $g:\L' \to \set{0,1}$, $x\in\L' \Leftrightarrow g(x) = 1$. Let $\varphi$ a bijection between $\set{0,1}$ and $\set{a,b}$ where $(a,b)\in\L\times\overline{\L}$. We have $x\in\L' \Leftrightarrow \varphi\circ g(x) = \varphi(a)$ ie. $x\in\L' \Leftrightarrow \varphi\circ g(x)\in\L$.
    
    So, $\textsc{DLogTime}$ is not close under polynomial-time \textsc{Karp} reduction.
\end{proof}

\begin{propo}
    $\textsc{Tautology}$ is $\NP$-hard.
\end{propo}
\begin{proof}
    Let $M$ a \textsc{Turing} machine with oracle $\textsc{Tautology}$. We will make that machine decide $SAT$ in polynomial time.
    
    This machine take a formula $\varphi$ compute the negation $\neg\varphi$ and test with its oracle if $\neg\varphi$ is a tautology. If it is not, there is a assignment which make $\neg\varphi$ false, so this assignment make $\varphi$ true and $\varphi$ is satisfiable. If $\neg\varphi$ is a tautology, $\varphi$ is not satisfiable.

    So $M$ decide $SAT$ and $\textsc{Tautology}$ is $\NP$-hard.
\end{proof}

\section{Difference of NP problems}

\begin{propo}\label{3:1}
    \textsc{ExactIndSet} is in DP.
\end{propo}
\begin{proof}
    Let $A$ be the set of all pairs $(G, k)$ such that $G$ has an independent set of size at least $k$, and let $B$ be the set of all pairs $(G, k)$ such $G$ has a independent set of size at least $k+1$. Then $\textsc{ExactIndSet}=A \setminus B$ and $A$ is in NP and $B$ is in NP. Hence by definition of $DP$, \textsc{ExactIndSet} is in $DP$.
\end{proof}

\begin{propo}\label{3:2}
    $\forall L \in DP$, $L$ is polynomial-time reducible to \textsc{ExactIndSet}.
\end{propo}
\begin{proof}
    \begin{lemma}
        $\textsc{IndSet} \geqslant_p 3-SAT$
    \end{lemma}
    \begin{proof}
        Suppose we have an instance $F$ of $3-SAT$ problem where $F=\bigwedge\limits_{i=1}^m C_i$ where $C_i$ is the disjunction of 3 variables. We note $x_1,\ldots,x_n$ the variables. We create the graph $G$ as follows:
        \begin{itemize}
            \item For each variable in each clause, create a vertex, which we will label with the name of the variable. Therefore there may be multiple vertices with the label $x_i$ or $\neg x_i$, if these variables appear in multiple clauses.
            \item For each clause, add an edge between the three vertices corresponding the variables from that clause.
            \item For all $i$, add an edge between every pair of vertices with one is labelled with $x_i$ and the other labelled with $x_i$.
        \end{itemize}
        
        There is a independent set of size $m$ in $G$ if and only if $F$ is satisfiable.    
    \end{proof}
    
    We note that this reduction from $3-SAT$ to \textsc{IndSet} took an instance $\varphi$ of $3-SAT$ consisting of $m$ clauses each of three literals and produced a graph $G_\varphi$ with $3 m$ vertices such that if $\varphi$ is satisfiable then the largest independent set in $G$ has m vertices, and if $G$ is unsatisfiable then the largest independent set of $G$ has at most $m - 1$ vertices. 
    
    Now suppose that $A$ is in $DP$. We want to show that $A \leqslant_p \textsc{ExactIndSet}$. By definition of $DP$, $A = L_1 \setminus L_2$ for $(L_1,L_2) \in NP^2$. Since $3-SAT$ is $NP$-complete, there are polytime functions $f_1$ , $f_2$ such that for $i = 1$ , $2$ and for all  $x \in \set{ 1 , 2 }^\star$ we have $x \in L_i \Leftrightarrow f_i ( x ) \in 3 SAT$. Hence for each fixed $x \in \set{ 1 , 2 }^\star$ , setting $\varphi_i = f_i(x)$, we have  $x \in L_i \Leftrightarrow \varphi_i$ is satisfiable Thus from the above reduction to \textsc{IndSet}, there is a polytime function which maps $x$ to a pair of graphs $G_1$, $G_2$ such that if $m_i$ is the number of clauses in $\varphi_i$, then for $i = 1$ , $2$, no independent set in $G_i$ has more than $m_i$ vertices, and $x \in  L_i \Leftrightarrow$ the largest independent set in $G_i$ has size $m_i$.
    
    Now we use the notation$ G\sqcup H$ for the disjoint union of graphs $G$ and $H$. That is, the vertices in $G\sqcup H$ are the disjoint union of those in $G$ and $H$, and similarly for the edges. Now let $G'_1=G_1\sqcup G_1$ Then a maximum independent set in $G'_1$ is the union of maximum independent sets in the two copies of $G_1$. Thus $x\in L_1\Rightarrow$ maximum independent set of $G'_1$ is $2m_1$ and $x\in \overline{L_1} \Rightarrow$ maximum independent set of $G'_1$ is $\leqslant 2m_1-2$. Now define $G'_2$ so that its vertices are those of $G_2$ together with $m_2-1$ new vertices, and the edges consist of those of $G_2$ together with an edge from each of the new vertices to every vertex of $G_2$. Then we have designed $G'_2$ so that no independent set can contain both vertices of $G_2$ and new vertices, so $x\in L_2 \Rightarrow$ maximum independent set of $G'_2$ is $m_2$, $x\in \overline{L_2} \Rightarrow$ maximum independent set of $G'_2$ is $m_2-1$. Now let $G_3=G'_1\sqcup G'_2$ and let $k= 2m_1+m_2-1$. To finish the proof that $A\leqslant_p DP$, it suffices to show that 
    
    \begin{lemma}
        $$x\in A\Leftrightarrow(G_3, k)\in DP$$
    \end{lemma}
    
    ($\Rightarrow$): Suppose $x\in A$. Then by (2)$x\in L_1 \cap L_2$, so by (3) and (6) we conclude the maximum independent set of $G_3=G'_1\sqcup  G'_2$ is $2m_1+m_2-1 =k$.
    
    ($\Leftarrow$): Suppose $x \not\in A$. There are three cases:
    \begin{itemize}
        \item $x\in L_1 \cap L_2\Rightarrow maxindset(G_3) = 2m_1+m_2> k$.
        \item $x\in L_1 \cap L_2\Rightarrow maxindset(G_3)\leqslant 2m_1+m_2-2< k$
        \item $x\in L_1 \cap L_2\Rightarrow maxindset(G_2) = 2m_1+m_2-3< k$
    \end{itemize}
\end{proof}

\begin{theoreme}
    \textsc{ExactIndSet} is $DP$-complete.
\end{theoreme}
\begin{proof}
    \textsc{ExactIndSet} is in $DP$ (proposition \ref{3:1}) and it is $DP$-hard (proposition \ref{3:2}).
\end{proof}

\section{Classes with exponential resources}

We name \textsc{BoundedHalt} the language of 3-tuples $\langle M,x,k\rangle$ where the machine $M$ halts on input $x$ in $k$ steps.

\begin{thm}
    \textsc{BoundedHalt} is $\Exp$-complete.
\end{thm}
\begin{proof}

\begin{lemma}\label{4:1}
    $\textsc{BoundedHalt} \in \Exp$.
\end{lemma}
\begin{proof}
Let $\langle M,x,k\rangle$ an instance of \textsc{BoundedHalt}.

We simulate $M$ on $x$ for $k$ steps and accepts if and only if $M$ halts and rejects otherwise. The running time is $m^{\cplx{1}}$. Let $n = \vert \langle\alpha,x,k\rangle \vert \geqslant \log k$. Therefore, the running time $\leqslant m^c = 2^{c\log m} \leqslant 2^{cn}$.
\end{proof}

\begin{lemma}\label{4:2}
    \textsc{BoundedHalt} is $\Exp$-hard.
\end{lemma}
\begin{proof}
For each language $\L \in \EXP$, we need to give polytime reduction from $\L$ to \textsc{BoundedHalt}. For a given language $\L \in \EXP$, we know there is a \textsc{Turing machine} $M_\L$ that decides A in time $g(n) \leqslant 2^{n^c}$ for a $c\in\NN$.

Let $f$ such that $f(w) = \langle M_\L, w, m\rangle$ where $m = 2^{|w|^c}$.

$f$ is polytime computable and $w \in \L \Rightarrow \langle M_\L, w, m\rangle \in \textsc{BoundedHalt}$ and $w \not\in \L \Rightarrow \langle M_\L, w, m\rangle \not\in \textsc{BoundedHalt}$
\end{proof}
    \textsc{BoundedHalt} is in $\Exp$ (lemma \ref{4:1}) and it is $\Exp$-hard (lemma \ref{4:2}).
\end{proof}

\begin{thm}
    $$\LogSpace=\P \Rightarrow \pSpace = \Exp$$
\end{thm}
\begin{proof}
Take an arbitrary $\L \in \Exp$, decided by a \textsc{Turing} machine in time $2^{n^c}$.

Let $L_{pad}:=\set{x\diamondsuit^{2^{\lvert x \rvert^c}} \mid x \in \L}$. We have $L_{pad} \in \P$ and, by the assumption $\P=\LogSpace$, we found that $L_{pad} \in \LogSpace$. So, there is some \textsc{Turing} machine $M_0$ deciding $L_{pad}$ in space $\log n$. $M_0$ can be used to decide $L$: on input $x$, simulate $M_0$ on the padded input $x\diamondsuit^{2^{\lvert x \rvert^c}}$ accept if and only if $M_0$ accepts.

The space we used on input $x$ ($\vert x\vert=n$) is the space used by $M_0$ on the padded input $x\diamondsuit^{2^{\lvert x \rvert^c}}$ of size $n + 2^{n^c}$, which is at most $\log\left(2^{n^c+1}\right) = n^c + 1$, a polynomial. Hence, we have that $\L \in \pSpace$. 

If we actually write the padded input $x\diamondsuit^{2^{\lvert x \rvert^c}}$ on the work tape, this would make the space usage exponential in $n$. But, we don't need to write entirely this padded input. We know what it looks like to the right of $x$. So we can simulate $M_0$ on the virtual padded input $x\diamondsuit^{2^{\lvert x \rvert^c}}$ by using a counter which tells the position on the tape of $M_0$. If $M_0$ enquires about the position to the right of $x$, we respond with the symbol $\diamondsuit$ (or the blank, if $M_0$ goes to the right of the entire padded input). Since the counter can assume the value at most $2^{n^c}$, we need at most $n^c$ bits for the counter. Thus, this new \textsc{Turing} machine uses space at most polynomial in $n$.
\end{proof}

\section{Downward self-reducibility}

\begin{notation}
    We note $\textsc{Dsr}$ the set of downward self-reducible languages.
\end{notation}

\begin{propo}
    $\P \subseteq \textsc{Dsr}$
\end{propo}
\begin{proof}
    Without query to the oracle, we can decide every language of $\P$.
\end{proof}

\begin{propo}
    $SAT \in \textsc{Dsr}$
\end{propo}
\begin{proof}
    Let $F$ a CNF-formula.
    $$
        F=\bigwedge\limits_{i=1}^m \bigvee\limits_{j=1}^{n_i} x_{i,j}^*
    $$
    
    where $x_{i,j}^*$ represent $x_{i,j}$ or $\neg x_{i,j}$ according with the indexes $i$ and $j$.
    
    $F \in SAT$ if and only if one of the $\left(x_{m,k} \wedge \bigwedge\limits_{i=1}^{m-1} \bigvee\limits_{j=1}^{n_i} x_{i,j}^* \right)_{k\in\llbracket 1,n_m\rrbracket} \in \left(CNF^{<\vert F \vert}\right)^{n_m    }$ is in SAT. We can test that in polynomial time.
\end{proof}
\begin{corol}
    $\NP$-complete $\subseteq \textsc{Dsr}$
\end{corol}
\begin{proof}
    Let $\L$ be a $\NP$ language. There exists a polynomial reduction showing that $SAT \leqslant_p \L$. The polynomial $p$ used is an element of $\ZZ[X]$ such that $\lim\limits_{n\to +\infty} p(n) = +\infty$. Consequently, there is a cofinite number of integers $n$ such that $p(n)<p(n+1)$. So, the downward self-reducibility as the same sense even after the reduction, except for a finite number of cases which can be hard-coded in a \textsc{Turing} machine. We conclude that $\L \in \textsc{Dsr}$.
\end{proof}

\begin{propo}
    $\textsc{Dsr} \subseteq \pSpace$
\end{propo}
\begin{proof}
    Let $\L \in \textsc{Dsr}$. Given an input $x$, we simulate the polytime computation that (with queries) decides $\L$, and recursively compute the answer to the each query as it is made. Since the recursive calls are all on strings shorter than $\vert x \vert$, we will eventually reach the base case in which we query strings of length 1. The program we are describing will simply have the answers to these (constant number of) length 1 queries hard-coded. The depth of the recursion is at most $\vert x \vert$, and at each level of recursion, we need to remember the state which requires space at most $p(\vert x\vert)$ where $p$ is a polynomial. This last point holds because the basic computation runs in polynomial time, and hence polynomial space. Thus the overall procedure runs in $\pSpace$.
\end{proof}

\section{Space hierarchy theorem}

\begin{thm}\label{6:1}
    There is a universal \textsc{Turing} machine $\mathcal{U}$ such that $\mathcal{U}(\alpha,x) = M_\alpha(x)$ and $\mathcal{U}$ halts on every input $(\alpha,x)$ after using at most $c S(|x|)$ space, where $S$ is the computation space of $M_\alpha$ and $c$ is some constant not depending on $x$.
\end{thm}

\begin{thm}
    For every space-constructible function $f:\NN \to \NN$,    
    $$
        \dSpace\left(o(f(n))\right) \subsetneq \dSpace(f(n))
    $$
\end{thm}
\begin{proof}
Let $f:\NN\to\NN$ be a space constructible function and $g:\NN\to\NN$ such that $g=o(f)$

We note $M$ a deterministic \textsc{Turing} machine. $M$'s computation:
\begin{itemize}
    \item $M$ writes $1^{f(n)}$ symbols on a work tape, where $n$ is the input's size.
    \item Simulate $\mathcal{U}(\alpha,(\alpha,x))$, and stops at most when the machine go out of the previous marks.
    \item If the computation of $\mathcal{U}(\alpha,(\alpha,x))$ is not terminated $M$ rejects.
    \item Otherwise, outputs the opposite result of $\mathcal{U}(\alpha,(\alpha,x))$
\end{itemize}

    The first step takes a space $f(n)$. The second step takes also a space $f(n)$ (thanks to the restriction). Finally, $M$ runs in space $\cplx{f(n)}$.

    We note $\L(M)$ the language recognized by $M$ ($\L(M)\in\dSpace(f(n))$) and assume that there is a deterministic machine $N$ such that $\L(M)= \L(N)$ and $N$ runs in space $g(n)$. We denote by $\alpha_0$ the code of $N$.

    According to the theorem \ref{6:1}, for every word $x$, $\mathcal{U}(\alpha_0,(\alpha_0,x))$ uses less than $c g(|(\alpha_0,x)|)$ space. And, by hypothesis, $c g(|(\alpha_0,x)|) \leqslant f(|(\alpha_0,x)|)$ for large enough $x$.

    Thus, for large enough $x$ we have $M(\alpha_0,x) = \neg\mathcal{U}(\alpha_0,(\alpha_0,x)) = \neg N(\alpha_0,x)$. This is a contradiction since we have assumed $\L(M)= \L(N)$.
  
  In conclusion, there exists no machine $N$ that runs in space $\cplx{g(n)}$ and recognizes $\L(M)$. Since we have proved $\L(M) \in \dSpace(f(n))$, we obtained $\dSpace (g(n)) \subsetneq \dSpace(f(n))$.
\end{proof}


\section{Polynomial hierarchy}

\begin{lemma}
    $\P^{SAT} = \Delta_2^P$
\end{lemma}
\begin{proof}
    Since $SAT$ is $\NP$-complete, all $\NP$ language is \textsc{Karp}-reducible to $SAT$, so $\P^{SAT} = \P^\NP$
\end{proof}

\begin{lemma}
    $\NP^{SAT} = \Sigma_2^P$
\end{lemma}
\begin{proof}
    $\NP^{SAT} = \NP^\NP$
\end{proof}

\begin{lemma}
    $\Delta_2^P = \Sigma_2^P \Rightarrow \Sigma_2^P \subseteq \Pi_2^P$
\end{lemma}
\begin{proof}
    Indeed $\Delta_2^P = \Sigma_2^P \cap \Pi_2^P$.
\end{proof}

\begin{lemma}
    $\Sigma_2^P \subseteq \Pi_2^P \Rightarrow \Sigma_2^P = \Pi_2^P$
\end{lemma}
\begin{proof}
    Let us assume $\Sigma_2^P \subseteq \Pi_2^P$. We have equivalently $\Sigma_2^P \subseteq co\Sigma_2^P$, so $\Sigma_2^P = co\Sigma_2^P$ and $\Sigma_2^P = \Pi_2^P$.
\end{proof}

\begin{thm}
    $\NP^{SAT}=\P^{SAT} \Rightarrow \Sigma_2^P = \Pi_2^P$
    
    ie. polynomial hierarchy collapse.
\end{thm}
\begin{proof}
    Just combine the previous lemmas.
\end{proof}

\end{document}
